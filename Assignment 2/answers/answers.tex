\documentclass[11pt]{article}

\usepackage[utf8]{inputenc}
\usepackage[margin=2.5cm]{geometry}
\usepackage{enumerate}
\usepackage{listings}
\usepackage{color}
\usepackage[hidelinks]{hyperref}
\usepackage{csquotes}

\setlength\parindent{0pt}
\setcounter{section}{3}

\definecolor{pblue}{rgb}{0.13,0.13,1}
\definecolor{pgreen}{rgb}{0,0.5,0}
\definecolor{pred}{rgb}{0.9,0,0}
\definecolor{pgrey}{rgb}{0.46,0.45,0.48}

\lstset{
	language=Java,
	tabsize=2,
	showspaces=false,
	showtabs=false,
	breaklines=true,
	showstringspaces=false,
	breakatwhitespace=true,
	commentstyle=\color{pgreen},
	keywordstyle=\color{pblue},
	stringstyle=\color{pred},
	basicstyle=\small\ttfamily,
	numbers=left,
	numberstyle=\tiny,
	numbersep=5pt
}

\title{Distributed Systems HS 2016\\Assignment 2}
\author{Markus Hauptner, Johannes Beck, Linus Fessler}
\date{\today}

\begin{document}
\maketitle

\section{Mini-Test}

\begin{enumerate}

\item
\begin{enumerate}[a)]

\item
It is defined as follows:
\begin{displayquote}
The Request-Line begins with a method token, followed by the Request-URI and the protocol version, and ending with CRLF. The elements are separated by SP characters. No CR or LF is allowed except in the final CRLF sequence.

\lstinline{Request-Line = Method SP Request-URI SP HTTP-Version CRLF}
\end{displayquote}

\item
\lstinline{Request-Line = GET 192.168.1.1:8080 HTTP/1.1}

\item
Representation: Interesting header-fields are \lstinline{Accept}, \lstinline{Accept-Charset}, \lstinline{Content-Length} and \lstinline{Content-Type}.

Caching: interesting header-fields \lstinline{Date}, \lstinline{Last Modified}, \lstinline{Cache-Control}, \lstinline{Expires} and \lstinline{Age}.

\end{enumerate}

\item
\begin{enumerate}[a)]

\item
The important classes are \lstinline{Socket} and \lstinline{ServerSocket}. The \lstinline{ServerSocket} implements a socket that servers can use to listen for and accept connections to clients. The server creates a \lstinline{new ServerSocket(port)} to listen at port \lstinline{port} and the clients create a \lstinline{new Socket(ip_address, port)} to connect to the server at \lstinline{ip_address:port}.

\item
The methods are \lstinline{getInputStream()} and \lstinline{getOutputStream()}.

Output streams are buffered and when the buffer is full---due to the client not being ready to receive any packets and not responding with acknowledgements so the server stops sending---but the current thread still wants to write to the stream, the stream will block the thread.

Input streams on the other hand will only block if the client didn't set the FIN flag on the last packet.

\end{enumerate}

\item
\begin{enumerate}[a)]

\item Incorrect, REST is not a protocol.
\item Incorrect, this is the definition of stateful.
\item Correct.
\item Incorrect, the server may send data in any format. HTML, XML and JSON are commonly used.

\end{enumerate}

\item
\begin{enumerate}[a)]

\item
Generally, WSDL can be used to describe network services in an XML document. In case of the SunSPOTWebservice, the URL to the document that describes its service is \\ \url{http://vslab.inf.ethz.ch:8080/SunSPOTWebServices/SunSPOTWebservice?wsdl}. The document can be simply retreived in a browser if the URL is known.

\item
They can be found in the WSDL document mentioned in a):
\begin{lstlisting}
<message name="getSpot">
	<part name="parameters" element="tns:getSpot"/>
</message>
<message name="getSpotResponse">
	<part name="parameters" element="tns:getSpotResponse"/>
</message>
\end{lstlisting}

\item
The \lstinline{soap:binding} tag defines \url{http://schemas.xmlsoap.org/soap/http} as transport protocol. This should be changed to, for example, \url{http://schemas.pocketsoap.com/soap/smtp} in order to use SMTP.

The port \lstinline{8080} in the \lstinline{soap:address} \url{http://vslab.inf.ethz.ch:8080/SunSPOTWebServices/SunSPOTWebservice} would probably be \lstinline{25} as this is the default port used by SMTP.

\end{enumerate}

\item
\begin{enumerate}[a)]

\item
Each emulated device will have its own router and will have assigned an IP address of \lstinline{10.0.2.15} behind its router.

\item
To the emulated device loopback interface.

\item
With IP address \lstinline{10.0.2.2}.

\item
By settings up port forwarding in adb with \lstinline{adb forward tcp:port1 tcp:port2} where port1 is the port on the development machine and port2 the port on the emulated device.

\end{enumerate}

\end{enumerate}

\end{document}
